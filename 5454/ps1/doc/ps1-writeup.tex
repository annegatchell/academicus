\documentclass[11pt, oneside]{article}   	% use "amsart" instead of "article" for AMSLaTeX format
\usepackage{geometry}                		% See geometry.pdf to learn the layout options. There are lots.
\geometry{letterpaper}                   		% ... or a4paper or a5paper or ... 
%\geometry{landscape}                		% Activate for for rotated page geometry
%\usepackage[parfill]{parskip}    		% Activate to begin paragraphs with an empty line rather than an indent
\usepackage{graphicx}				% Use pdf, png, jpg, or eps� with pdflatex; use eps in DVI mode
								% TeX will automatically convert eps --> pdf in pdflatex		
\usepackage{amssymb}

\title{Problem Set 1}
\author{Anne Gatchell}
\date{Due February 4, 2013}							% Activate to display a given date or no date

\renewcommand\thesubsection{\Alph{subsection}}

\begin{document}
\maketitle
\section{}
For reference: 
$1 day *(24 hrs/1 day)*(3600s/1 hr) *(10^{6}/1s) = 8.64 x 10^{4} \mu s$


\subsection{}
In this case, Acme should pay professor Flitwick.\\
$ n = 41$\\
$f(n) = 1.99^{n}$ \\
$g(n) = n^{3}$ \\
$t = 17 days = 1.4688 x 10^{12} \mu s$\\\\
Without Flitwick, it takes $ f(41) = 1.99^{41} = 1.79 x 10^{12} \mu s$\\
With Flitwick it takes $ t + g(n) = 1.47 x 10^{12} \mu s + 41^{3} = 1.46 x 10^{12}\mu s$\\\\
It will take Flitwick less time to spend 17 days working and then running his program than it will take to just run the program by itself:\\\\
Flitwick's alg is $17 days, 0.069s$\\
The other lag will take $20 days, 17 hours, 21 min, 47.45 seconds$
\subsection{}
In this case, Acme should not pay professor Flitwick.\\
$ n = 10^{6}$\\
$f(n) = n^{2.00}$ \\
$g(n) = n^{1.99}$ \\
$t = 2 days = 2*8.64 x 10^{10} \mu s = 1.728 x 10^{11} \mu s$\\\\
Without Flitwick, it takes $ f(10^{6}) = (10^{6})^{2.00} = 10^{12} \mu s = \\11.574 days = 11 days, 13 hours, 46 minutes, 40 seconds$\\\\
With Flitwick it takes $ t + g(n) = 2 days + (10^{6})^{1.99} = 1.73 x 10^{11} \mu s + 8.71 x 10^{11}\mu s = 1.044 x 10^{12} \mu s$\\
$1.044x 10^{12}/8.64 x 10^{10} \mu s = 12.08 days = 12 days, 1 hr, 56 min, 3.6s$\\\\
It is a close call, but it will be faster to just run the original algorithm and not pay Flitwick.
\section{}%2
\section{}%3
\subsection{}%a
Is there a c for which $0 \le 2^{nk} \le c2^{n}$  for $ k > 1$?\\\\
Dividing both sides by $2^{n}$:\\
$(2^{n})^{k-1} \le c$\\\\
No. There is no constant c that is greater or equal to $(2^{n})^{k-1}$ for sufficiently large n.
\subsection{}%b
Is there a c for which $0 \le 2^{n+k} \le c2^{n}$  for $ 0 \le k \le c$ (some positive constant)?\\\\
$2^{n}*2^{k} \le c2^{n}$\\
Dividing both sides by $2^{n}$:\\
$2^{k} \le c$for 
Yes. For $n_{o} = 0, c \ge 2^{k}$ where $k \ge 0$.
\section{}%4
\subsection{}%a
\subsection{}%b
Final order:
$1$, $n^{1/lg n}$, $2^{lg*n}$, $(\sqrt{2})^{lg n}$, $n$, $n lg n$, $lg(n!)$, $n^{2}$, $(3/2)^{n}$, $e^{n}$, $(lg n)!$, $n!$\\\\

Both $n lg n$, $lg(n!)$ are in the $O(n lg n)$ equivalence class.


\end{document}  